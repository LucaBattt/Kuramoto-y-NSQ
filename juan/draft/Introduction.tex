\documentclass[draft.tex]{subfiles}
\graphicspath{{\subfix{../images/}}}
\begin{document}
La mayoría de los organismos en la tierra evolucionaron en un ambiente cíclico debido
al movimiento de rotación terrestre, el cual se expresa en diferentes condiciones de luz,
temperatura, acceso a la alimentación, etc. Estas condiciones ambientales cíclicas fueron
claves para que los organismos desarrollaran mecanismos de temporización endógenos que
les permiten adaptarse a cambios periódicos en el ambiente y organizar temporalmente los
procesos biológicos internos, produciendo oscilaciones genéticas, metabólicas, fisiológicas,
hormonales, conductuales y sociales. Se denominan ritmos biológicos a estos procesos
rítmicos observados en bacterias, plantas y hasta en los humanos. Los ritmos biológicos
pueden ser clasificados de acuerdo a sus periodicidades que van desde segundos hasta años.
Los ritmos circadianos son un caso particular de ritmos biológicos, donde las oscilaciones
presentan periodos cercanos a las 24 horas. Estos ritmos se manifiestan en los organismos
como patrones diarios de alimentación, presión sanguínea, regulación de hormonas y
ciertas funciones celulares fundamentales como la expresión génica y la transcripción de
proteínas. Los mecanismos de temporización endógenos que generan ritmos circadianos
en los organismos son comúnmente denominados relojes circadianos.
Durante mucho tiempo los investigadores se preguntaban como era posible que los
mamíferos se anticiparan a la luz del día o la oscuridad de la noche, y en qué parte del
organismo se encontraba dicho sistema interno de temporización.
Numerosos experimentos demostraron que los mamíferos poseen estos relojes circadi-
anos en la mayoría de las células constituyentes de tejidos u órganos periféricos como el
hígado, riñones, pulmones, etc. Se demostró que existe un conjunto de genes llamados
genes reloj que interactúan entre sí mediante las proteínas que esos genes codifican,
formando circuitos de retroalimentación transcripcionales-traduccionales (TTFL por sus
siglas en inglés). Estos relojes circadianos moleculares en las células se sincronizan entre
sí a nivel de tejido y, a su vez, existe una sincronización entre tejidos y órganos, la cual es
orquestada por una región del cerebro, el Núcleo Supraquiasmatico (NSQ). El NSQ esta
compuesto por cerca de 20000 células (incluyendo neuronas y glías), las cuales exhiben un
comportamiento oscilatorio endógeno en la expresión génica, su metabolismo y su actividad
eléctrica, entre otros procesos. Estas neuronas sincronizan sus oscilaciones dando lugar a
la aparición de ritmos circadianos robustos en el NSQ.
El hecho de que las células del NSQ se sincronicen entre sí sugiere la existencia de
una comunicación permanente entre ellas, es decir que, las neuronas y glias se encuentran
acopladas entre si formando una red de osciladores circadianos celulares cuya estructura
topológica se ha intentado determinar a través de experimentos \textit{in vitro}, sin resultados concluyentes unívocos.
Estas características sobre el NSQ abren la puerta a un estudio del sistema desde el
punto de vista de la física a través de modelos de osciladores autónomos acoplados en
redes complejas de distinta topología para caracterizar el proceso de sincronización entre
osciladores y también con una perturbación externa en función de la topología de la red
\end{document}